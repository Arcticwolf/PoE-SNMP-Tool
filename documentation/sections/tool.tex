\section{PoE SNMP Monitoring Tool}
\label{sec:tool}

\subsection{Architektur}

Das PoE-Tool besteht aus 2 Komponenten.

\begin{description}
  \item [\textbf{Backend}] Die Backend-Komponente erzeugt kontinuierlich
  Messungen für alle definierten Switches durch. Die Zeitintervalle zwischen den Messungen
  werden mittels config-file angegeben. Die Messungen werden weiters in der
  Datenbank abgelegt. Zusätzlich bietet das Backend eine Schnittstelle an welche
  es erlaubt Abfragen auf die gesammelten Measurements zu machen.
  \item [\textbf{GUI}] Die GUI-Komponente nimmt Benutzeranfragen entgegen und
  fragt die darzustellenden Daten vom Backend ab. Das Backend liefert jedoch nur
  Messungen. Die Gruppierung und Konvertierung der Daten in ein übersichtliches
  von Menschen verwertbares Datenmaterial wird ebenfalls von der GUI
  durchgeführt.  
\end{description}


\begin{figure}[h]
    \centering
    \leavevmode
    \includegraphics[width=1.0\linewidth]{figures/architecture.jpeg}
    \caption{Software-Architektur PoE}
    \label{fig:Software-Architektue PoE}
\end{figure}

\subsection{Konfiguration}

Das PoE-Tool wird über die Konfigurationsdatei
\textbf{config.properties} angepasst. Diese enthält folgende Optionen:

\begin{description}
  \item [\textbf{measurement.interval}] Dieser Wert gibt die Zeit die zwischen
  Messungen verstreicht in ms an.
  \item [\textbf{distribution.slots}] Dieser Wert gibt in wieviele Zeitslots ein
  Messintervall unterteilt werden soll, um eine gleichmäßigere Verteilung an
  SNMP-Anfragen an die einzelnen Switches zu ermöglichen.
  \item [\textbf{data.retriever.impl}] Dieser Wert gibt an welche Datenquelle
  für die GUI verwendet werden soll. Es gibt 2 Datenquellen. Einerseits eine
  Test-Datenquelle für die GUI
  \textit{cn.poe.group1.collector.DummyDataRetriever} und zusätzlich die
  Datenquelle welche die tatsächlichen Nutzdaten liefert \textit{cn.poe.group1.collector.DataRetrieve}
\end{description}

\subsection{Verwendung}

Durch das Starten des PoE-Tool beginnt dieses automatisch Messungen für alle
definierten Switches zu erzeugen. Durch Auswahl eines entsprechendes Switches in
der linken Tabelle  können für diesen die Nutzdaten angezeigt werden.

Wichtig ist hier die GUI zeigt immer die Messdaten aus einem bestimmten Zeitraum
an. Dieser kann im PoE-Hauptfenster rechts oben definiert werden. Um die Daten
zu aktualisieren klickt man einfach auf \textbf{Reload}.

\subsubsection{Switch}

Durch Auswahl eines Switches in der linken Tabelle erhält man einen Überblick
über den Switch. Auf der Registerkarte Switch im rechten Teil des Fensters
erhält man einen Überblick den zeitlichen Verlauf der Messungen am Switch.

\begin{figure}[h]
    \centering
    \leavevmode
    \includegraphics[width=1.0\linewidth]{figures/screenshot2.jpg}
    \caption{Software-Architektur PoE}
    \label{fig:Architecture-PoE}
\end{figure}

Durch einen Klick auf \textbf{Add Switch} in der Menüzeile kann man einen neuen
Switch definieren welcher überwacht werden soll. Dazu müssen lediglich die Daten
im Popup ausgefüllt und bestätigt werden. 

 \begin{figure}[h]
    \centering
    \leavevmode
    \includegraphics[scale=0.5]{figures/screenshot3.jpg}
    \caption{Software-Architektur PoE}
    \label{fig:NeuerSwitch-PoE}
\end{figure}

Zur Editierung und Löschung von Switches macht man einfach einen Rechtsklick auf
den entsprechenden Switch in der linken Tabelle. Danach taucht ein Kontextmenü
auf welches 2 Einträge hat \textbf{Edit Switch} (Editieren von Switches) und
\textbf{Delete Switch} (Löschen von Switches).

\subsubsection{Ports}

Auf der Registerkarte Port im rechten Teil des Fensters können die Daten über
die einzelnen Ports eines Switchs betrachtet werden. Die Tabelle enthält dabei
einen Mittwert aus den Messungen über die einzelnen Parameter. Der Chart enthält
einen graphischen Verlauf der einzelnen Parametern.

\begin{figure}[h]
    \centering
    \leavevmode
    \includegraphics[width=1.0\linewidth]{figures/screenshot1.png}
    \caption{Ports PoE}
    \label{fig:Ports-PoE}
\end{figure}

\subsubsection{Messungen}

Messungen werden automatisch durch das Starten des PoE-Tools erzeugt. Die
Parameter für die Messmethode werden mittels config-file konfiguriert. Siehe Abschnitt Konfiguration.

\subsubsection{CSV Export}

Durch einen Klick auf den Button \textbf{Export} können sämtliche Messungen im
angegebenen Zeitraum welcher definiert ist in ein csv-file export werden. Dazu
muss dann lediglich im auftauchenden Datei-Dialog der entsprechende Ort und der
Dateiname gewählt werden.
